\documentclass{article}
% Some basic packagesLLuu
\usepackage[utf8]{inputenc}
\usepackage{spverbatim}
\usepackage[margin=1.2in]{geometry}
\usepackage{textcomp}
\usepackage{url}
\usepackage{graphicx}
\usepackage{float}
\usepackage{algorithm2e}
% \usepackage{algorithm}
\usepackage{enumitem}
\usepackage{standalone}
\usepackage{tcolorbox}
\usepackage{wrapfig}
% \usepackage{svg}
% \usepackage{svg-inkscape} 

\graphicspath{{./figures}}


%color settings
\usepackage{xcolor}
% \definecolor{gruvbgdark}{HTML}{1d2021}
% \definecolor{gruvtextdark}{HTML}{ebdbb2}
% \definecolor{gruvbglight}{HTML}{f9f5d7}
% \definecolor{gruvtextlight}{HTML}{3c3836}
% \definecolor{NavyBlue}{HTML}{266bbd}
% \definecolor{RawSienna}{HTML}{94330e}
% \definecolor{ForestGreen}{HTML}{149b52}
% \pagecolor{gruvbgdark}
% \color{gruvtextdark}

\definecolor{GBBlue}{HTML}{458588}
\definecolor{GBGreen}{HTML}{98971A}
\definecolor{GBRed}{HTML}{CC241D}
\definecolor{GBYellow}{HTML}{D79921}
\definecolor{GBPurple}{HTML}{B16286}
\definecolor{GBAqua}{HTML}{689D6A}
\definecolor{GBOrange}{HTML}{D65D0E}
\definecolor{GBGray}{HTML}{A89984}
\definecolor{GBDarkGray}{HTML}{3C3836}
\definecolor{GBLightGray}{HTML}{EBDBB2}
\definecolor{GBBackground}{HTML}{282828}
\definecolor{GBForeground}{HTML}{FBF1C7}
\pagecolor{GBBackground}
\color{GBForeground}
% Hide page number when page is empty
\usepackage{emptypage}
\usepackage{subcaption}
\usepackage{multicol}

% Math stuff
\usepackage{amsmath, amsfonts, mathtools, amsthm, amssymb}
% Fancy script capitals
\usepackage{mathrsfs}
\usepackage{cancel}

% Bold math
\usepackage{bm}

%Algorithm setup
\RestyleAlgo{algoruled}
% Some shortcuts
\newcommand\N{\ensuremath{\mathbb{N}}}
\newcommand\R{\ensuremath{\mathbb{R}}}
\newcommand\Z{\ensuremath{\mathbb{Z}}}
\renewcommand\O{\ensuremath{\emptyset}}
\newcommand\Q{\ensuremath{\mathbb{Q}}}
\newcommand\C{\ensuremath{\mathbb{C}}}
\newcommand\B{\ensuremath{\mathbb{B}}}

%Make implies and impliedby shorter
\let\implies\Rightarrow
\let\impliedby\Leftarrow
\let\iff\Leftrightarrow


% Add \contra symbol to denote contradiction
% \usepackage{stmaryrd} % for \lightning
% \newcommand\contra{\scalebox{1.5}{$\lightning$}}

\let\phi\varphi

% Command for short corrections
% Usage: 1+1=\correct{3}{2}

\definecolor{correct}{HTML}{009900}
\newcommand\correct[2]{\ensuremath{\:}{\color{red}{#1}}\ensuremath{\to }{\color{correct}{#2}}\ensuremath{\:}}
\newcommand\green[1]{{\color{correct}{#1}}}

% horizontal rule
% \newcommand\hr{
%     \noindent\rule[0.5ex]{\linewidth}{0.5pt}
% }

% hide parts
\newcommand\hide[1]{}

% Environments
\makeatother

% For box around Definition, Theorem, \ldots
% theorems
\usepackage{thmtools}
\usepackage[framemethod=TikZ]{mdframed}
\mdfsetup{skipabove=1em,skipbelow=1em, innertopmargin=5pt, innerbottommargin=6pt}

\theoremstyle{definition}

\makeatletter

\declaretheoremstyle[headfont=\bfseries, bodyfont=\normalfont, numbered=no]{idea}

\declaretheoremstyle[
	headfont=\bfseries\color{ForestGreen!70!black}, bodyfont=\normalfont,
    postheadspace=\newline,
	mdframed={
			linewidth=2pt,
			rightline=false, topline=false, bottomline=false,
			linecolor=ForestGreen, backgroundcolor=ForestGreen!5,
		},
    numberwithin=section
]{thmgreenbox}

\declaretheoremstyle[
	headfont=\bfseries\color{NavyBlue!70!black}, bodyfont=\normalfont,
    postheadspace=\newline,
	mdframed={
			linewidth=2pt,
			rightline=false, topline=false, bottomline=false,
			linecolor=NavyBlue, backgroundcolor=NavyBlue!5,
		},
    numberwithin=section
]{thmbluebox}
\declaretheoremstyle[
	headfont=\bfseries\color{NavyBlue!70!black}, bodyfont=\normalfont,
	mdframed={
			linewidth=2pt,
			rightline=false, topline=false, bottomline=false,
			linecolor=NavyBlue, backgroundcolor=NavyBlue!5,
		},
    numberwithin=section
]{thmclaimbox}

\declaretheoremstyle[
	headfont=\bfseries\color{NavyBlue!70!black}, bodyfont=\normalfont,
	mdframed={
			linewidth=2pt,
			rightline=false, topline=false, bottomline=false,
			linecolor=NavyBlue
		},
    numberwithin=section
]{thmblueline}

\declaretheoremstyle[
	headfont=\bfseries\color{RawSienna!70!black}, bodyfont=\normalfont,
    postheadspace=\newline,
	mdframed={
			linewidth=2pt,
			rightline=false, topline=false, bottomline=false,
			linecolor=RawSienna, backgroundcolor=RawSienna!5,
		},
    numberwithin=section
]{thmredbox}

\declaretheoremstyle[
	headfont=\bfseries\color{RawSienna!70!black}, bodyfont=\normalfont,
    postheadspace=\newline,
	mdframed={
			linewidth=2pt,
			rightline=false, topline=false, bottomline=false,
			linecolor=RawSienna, backgroundcolor=RawSienna!5,
		},
	qed=\qedsymbol,
    numberwithin=section
]{thmproofbox}

\declaretheoremstyle[
	headfont=\bfseries\color{NavyBlue!70!black}, bodyfont=\normalfont,
	numbered=no,
	mdframed={
			linewidth=2pt,
			rightline=false, topline=false, bottomline=false,
			linecolor=NavyBlue, backgroundcolor=NavyBlue!1,
		},
]{thmexplanationbox}

\declaretheorem[style=thmgreenbox, name=Definition]{definition}
\declaretheorem[style=thmbluebox, numbered=no, name=Problem Statement]{stmt}
\declaretheorem[style=thmbluebox, numbered=no, name=Answer]{ans}
\declaretheorem[sibling=definition, style=thmredbox, name=Corollary]{corollary}
\declaretheorem[style=thmbluebox, numbered=no, name=Idea]{idea}
\declaretheorem[style=thmbluebox,  name=Proposition]{prop}
\declaretheorem[style=thmredbox, name=Theorem]{theorem}
\declaretheorem[style=thmproofbox, name=Proof]{prf}
\declaretheorem[style=thmredbox, name=Lemma]{lemma}
\declaretheorem[style=thmclaimbox, numbered=no, name=Claim]{claim}
\declaretheorem[numbered=no, style=thmexplanationbox, name=Proof]{explanation}
\declaretheorem[numbered=no, style=thmproofbox, name=Proof]{replacementproof}
\declaretheorem[style=thmgreenbox,  numbered=no, name=Task]{ex}
\declaretheorem[style=thmbluebox,  numbered=no, name=Example]{eg}
\declaretheorem[style=thmblueline, numbered=no, name=Remark]{remark}
\declaretheorem[style=thmblueline, numbered=no, name=Note]{note}

\renewenvironment{proof}[1][\proofname]{\begin{replacementproof}}{\end{replacementproof}}
\AtEndEnvironment{eg}{\null\hfill$\diamond$}%

\newtheorem*{uovt}{UOVT}
\newtheorem*{notation}{Notation}
\newtheorem*{previouslyseen}{As previously seen}
\newtheorem*{problem}{Problem}
\newtheorem*{observe}{Observe}
\newtheorem*{property}{Property}
\newtheorem*{intuition}{Intuition}

\newcommand{\oefening}[1]{%
	\def\@oefening{#1}%
	\subsection*{Oefening #1}
}

\newcommand{\suboefening}[1]{%
	\subsubsection*{Oefening \@oefening.#1}
}


% \lecture starts a new lecture (les in dutch)
%
% Usage:
% \lecture{1}{di 12 feb 2019 16:00}{Inleiding}
%
% This adds a section heading with the number / title of the lecture and a
% margin paragraph with the date.

% I use \dateparts here to hide the year (2019). This way, I can easily parse
% the date of each lecture unambiguously while still having a human-friendly
% short format printed to the pdf.

% \usepackage{xifthen}
% \def\testdateparts#1{\dateparts#1\relax}
% \def\dateparts#1 #2 #3 #4 #5\relax{
% 	\marginpar{\small\textsf{\mbox{#1 #2 #3 #5}}}
% }

% \def\@lecture{}%
% \newcommand{\lecture}[3]{
% 	\ifthenelse{\isempty{#3}}{%
% 		\def\@lecture{Lecture #1}%
% 	}{%
% 		\def\@lecture{Lecture #1: #3}%
% 	}%
% 	\subsection*{\@lecture}
% 	% \marginpar{\small\textsf{\mbox{#2}}}
% }

\usepackage{listings}

\definecolor{dkgreen}{rgb}{0,0.6,0}
\definecolor{gray}{rgb}{0.5,0.5,0.5}
\definecolor{mauve}{rgb}{0.58,0,0.82}

\lstset{frame=single,
  language=c,
  aboveskip=3mm,
  belowskip=3mm,
  showstringspaces=false,
  columns=flexible,
  basicstyle={\small\ttfamily},
  numbers=none,
  numberstyle=\tiny\color{GBForeground},
  keywordstyle=\color{GBRed},
  commentstyle=\color{GBGreen},
  stringstyle=\color{GBAqua},
  breaklines=true,
  breakatwhitespace=true,
  tabsize=2
}



% These are the fancy headers
\usepackage{fancyhdr}
\pagestyle{fancy}

% LE: left even
% RO: right odd
% CE, CO: center even, center odd
% My name for when I print my lecture notes to use for an open book exam.
\fancyhead[LE,RO]{Kristian Sørdal}

\fancyhead[RO,LE]{INF236 - Parallel Programmering} % Right odd,  Left even
% \fancyhead[RE,LO]{\leftmark}          % Right even, Left odd
\fancyhead[RE,LO]{Kristian Sørdal}          % Right even, Left odd

% \fancyfoot[RO,LE]{\thepage}  % Right odd,  Left even
% \fancyfoot[RE,LO]{Kristian Sørdal}          % Right even, Left odd
% \fancyfoot[C]{\leftmark}     % Center

\makeatother

% Todonotes and inline notes in fancy boxes
\usepackage{todonotes}
\usepackage{tcolorbox}

% Make boxes breakable
\tcbuselibrary{breakable}

% Figure support as explained in my blog post.
\usepackage{import}
\usepackage{xifthen}
\usepackage{pdfpages}
\usepackage{transparent}
\newcommand{\incfig}[2][1]{%
	% \begin{center}
	\def\svgwidth{#1\columnwidth}
	\import{./figures/}{#2.pdf_tex}
	% \end{center}
}
% Fix some stuff
% %http://tex.stackexchange.com/questions/76273/multiple-pdfs-with-page-group-included-in-a-single-page-warning
\pdfsuppresswarningpagegroup=1
\author{Kristian Sørdal}

\usepackage{pgfplots}
\usepackage{pgfplotstable}
\pgfplotsset{compat=1.17}

% 80 32.802406 7.405352 9.848995 22.588494

\title{INF236 - Assignment 2}

\newcommand{\Barrier}{\textbf{barrier}}

\begin{document}
\maketitle
\newpage
\tableofcontents
\newpage
\section{Introduction}
In this assignment, we are experimenting with two different strategies for paralellizing breadth first search. There will be two different parallel algorithms implemented.
\medskip

Breadth first search or just BFS, is a classical graph exploration algorithm, where the input is given as a graph \( G\left( V,E \right) \), and some starting node \( n \in V \). We can then use BFS to figure out various properties about the graph, including (but not limited to) finding the shortest path between the starting node and some other node, the path used to get from one node to another node, and the amount of nodes that are reachable from a given node, and consequently whether or not the graph is connected. 
\medskip

BFS explores the graph layer by layer, which means that from the starting node, all of the neighbours of this node are added to a queue, and the distance \( d \) to these nodes will be equal to 1 plus the distance from the starting node. This process is repeated for these nodes neighbours once again, and so on until there are no more nodes to explore (the queue is empty). We also need to keep track of which nodes we have seen, and if we encounter a node we have already seen, we need to check if we have found a shorter distance to this node, and update that nodes distance accordingly.

\section{Parallel BFS}
Two different verions of parallel BFS have been implemented for this assignment, with two different approached to the parallelization.

\subsection{First implementation}
When parallelizing BFS, we can observe that nodes are found layer by layer. So given a start node with distance 0, we will discover all its neighbours. These nodes will have distance 1 from the start node, their neighbours again will have distance 2, and so on. This lends itself nicely to paralellizing. If we maintain all nodes found in a given layer in some queue, we can distribute this queue amongst the different threads, using the OpenMP construct \texttt{\#pragma omp for}. Then, every thread can work locally on discovering neighbours of these nodes, and maintaining these found neighbours in a private list. We don't care about which parent a given node points to, as long as the distance from the start node is correct. This means that the first thread that discoveres a node without a parent pointer, will set the parent pointer of that node, even if some other node in the same layer is also a neighbour of this node.
\medskip

After all nodes in the current layer have been discovered, we can for each thread, copy the private discovered list back into the global layer queue, and continue on, until there are no more nodes to discover.
\medskip

An outline of this approach is given below. Note that this routine is called from within a parallel environment, where the parameters to the function are shared between all threads.

\begin{algorithm}[H]
    \caption{Parallel BFS - First approach}
    \SetAlgoVlined
    \SetKwInOut{Input}{Input}
    \SetKwInOut{Output}{Output}
    \Input{\textit{G(V,E)} - Graph, \textit{dist} - Distance Array, \textit{p} - Parent Array, \textit{S} - Global Queue, \textit{T} - Temporary global queue }
    \Output{\textit{dist} and \textit{p}, populated with distances and parent of all nodes\newline}
    layerSize \( \leftarrow 1 \)\\
    numDiscovered \( \leftarrow 0 \)\\
    discovered \( \leftarrow \) list of size \( n \)\\
    p[1] \( \leftarrow \) 0 \\
    dist[1] \( \leftarrow 0 \)  \\
    S[0] \( \leftarrow 0\) \\

    \While{layerSize \( \neq 0\)}{
        \Barrier\\
        \ForPar{ i = 0 \KwTo layerSize}{
            \For{ \( v \in \text{neighbours}(S[i]) \)}{
            \If{p[v] == -1}{
                dist[v] \( \leftarrow \) dist[S[i]] + 1 \\
                p[v] \( \leftarrow \) S[i] \\
                discovered[numDiscovered++] \( \leftarrow \) v \\
            }
            }
        }
        T[tid] \( \leftarrow \) numDiscovered \\
        \Barrier\\
        layerSize \( \leftarrow \) T[0] \\
        offset \( \leftarrow 0 \)\\
        \For{ i = 1 \KwTo numThreads}{
            \If{\( i == tid \)}{
                offset \( \leftarrow \) layerSize\\
            }
            layerSize += T[i] \\
        }

        \If{numDiscovered \( > 0 \)}{
            S[offset : numDiscovered] = discovered\\
            numDiscovered \( \leftarrow 0 \)\\
        }
    }
\end{algorithm}

\subsection{Second implementation}
For the second implementation, the strategy is as follows. First, we will perform BFS sequentially until some condition is reached. This condition can for example be: \textit{perform sequential BFS until the layer discovered contains \( n \cdot \#threads \) nodes}.
\medskip

After this, we distribute the newly discovered nodes into a local queue for each thread as evenly as possible. Then each thread will continue discovering all nodes with distance at most \( k \) away from the originally distributed layer. 
\medskip

Then, we put all nodes in this layer into the global queue, and redistribute the nodes amongst the ranks again, as evenly as possible. This is repeated until there are no more nodes to discover. A general outline of how this algorithm is implemented is given below.

\begin{algorithm}[H]
    \caption{Parallel BFS - Second approach}
    \SetAlgoVlined
    \SetKwInOut{Input}{Input}
    \SetKwInOut{Output}{Output}
    \Input{\textit{G(V,E)} - Graph, \textit{dist} - Distance Array, \textit{p} - Parent Array, \textit{S} - Global Queue, \textit{T} - Temporary global queue }
    \Output{\textit{dist} and \textit{p}, populated with distances and parent of all nodes\newline}
    stepsPerRound \( \leftarrow 4 \)\\
    threadID \( \leftarrow \) index of the current thread\\
    layerSize \( \leftarrow 1 \)\\
    localLayerSize \( \leftarrow 0 \)\\
    localQueue \( \leftarrow \) list of size \( n \)\\
    discovered \( \leftarrow \) list of size \( n \)\\
    numDiscovered \( \leftarrow 0 \)\\
    \phantom{a}\\
    p[1] \( \leftarrow \) 0 \\
    dist[1] \( \leftarrow 0 \)  \\
    S[0] \( \leftarrow 0\) \\
    \phantom{a}\\
    Perform sequential BFS until \textit{\#threads} nodes are found, store found nodes at last layer in S, and layer size in T[0]\\

    layerSize \( \leftarrow \) T[0] \\
    \phantom{a}\\

    Distribute nodes in S to local queues\\
    \phantom{a}\\

    \While{\( layerSize \neq 0 \)}{
        \For{\( i = 0 \) \KwTo stepsPerRound}{
            \Barrier\\
            \For{\( j = 0 \) \KwTo localLayerSize}{
                \For{\( v \in \text{neighbours}(S[j]) \)}{
                    \If{p[v] == -1}{
                        dist[v] \( \leftarrow \) dist[S[j]] + 1 \\
                        p[v] \( \leftarrow \) S[j] \\
                        discovered[numDiscovered++] \( \leftarrow \) v \\
                    }
                }
            }
            Swap pointers of \textit{localQueue} and \textit{discovered}\\
            localLayerSize \( \leftarrow \) numDiscovered\\
            numDiscovered \( \leftarrow 0\)\\
        }
        T[tid] \( \leftarrow\) localLayerSize \\
        \Barrier

        layerSize \( \leftarrow \) T[0] \\
        offset \( \leftarrow 0 \)\\
        \For{\( i = 1 \)\KwTo numThreads}{
            \If{\( i == tid \)}{
            offset \( \leftarrow \) layerSize\\
            }
            layerSize += T[i] \\
        }
        \phantom{a}\\
        \Barrier
        Copy localQueue to S starting at offset
        \Barrier
        \phantom{a}\\

        Redistribute S evenly into local queues
    }
\end{algorithm}


\section{Experiments}
\newpage
    \pgfplotstableread{
    T roadusa rggn222s0 delaunayn24 hugebubbles-00020
    1 1.734203 1.081684 1.783102 3.489007
    2 1.200486 0.606074 1.025448 1.882132
    4 0.788950 0.335648 0.627425 0.974213
    8 0.461280 0.188359 0.355146 0.523627
    16 0.327205 0.124455 0.236014 0.373559
    32 0.357365 0.126939 0.246382 0.392521
    64 0.699687 0.214437 0.498225 0.856313
    }\timepbfs

    \pgfplotstablecreatecol[create col/expr={1.739677/\thisrow{roadusa}}]{roadusaspeedup}{\timepbfs}
    \pgfplotstablecreatecol[create col/expr={1.042576/\thisrow{rggn222s0}}]{rggn222s0speedup}{\timepbfs}
    \pgfplotstablecreatecol[create col/expr={1.549098/\thisrow{delaunayn24}}]{delaunayn24speedup}{\timepbfs}
    \pgfplotstablecreatecol[create col/expr={4.052872/\thisrow{hugebubbles-00020}}]{hugebubblesspeedup}{\timepbfs}

    \begin{figure}[H]
       \centering 
        \begin{minipage}{0.45\textwidth}
            \begin{tikzpicture}[scale=0.85]
    \begin{axis}[
        xlabel={Number of Threads},
        ylabel={Execution Time (Seconds)},
        xmin=1, xmax=64,
        xticklabels={1,2,4,8,16,32,64},
        % xticklabels from table={\testdata}{T},
        xmode=log,
        log basis x={2},
        ymin=0, % Adjust based on your data
        legend pos=north west,
        ymajorgrids=true,
        xmajorgrids=true,
        grid style=dashed,
        legend style={cells={anchor=west}, legend pos=north east,font=\tiny},
    ]

    % Plot for roadusa
    \addplot[color=GBBlue, mark=*] table [x=T, y=roadusa] {\timepbfs};
    \addlegendentry{road\_usa}

    % Plot for rggn222s0
    \addplot[color=GBRed, mark=square*] table [x=T, y=rggn222s0] {\timepbfs};
    \addlegendentry{rgg\_n\_2\_22\_s0}

    % Plot for delaunayn24
    \addplot[color=GBGreen, mark=triangle*] table [x=T, y=delaunayn24] {\timepbfs};
    \addlegendentry{delaunay\_n24}

    % Plot for hugebubbles-00020
    \addplot [color=GBYellow, mark=pentagon*] table [x=T, y=hugebubbles-00020] {\timepbfs};
    \addlegendentry{hugebubbles-00020}

    \end{axis}
    \end{tikzpicture}
\end{minipage}\hspace{0.5cm}
        \begin{minipage}{0.45\textwidth}
\begin{tikzpicture}[scale=0.85]
    \begin{axis}[
        xlabel={Number of Threads},
        ylabel={Speedup},
        xmin=1, xmax=64,
        xticklabels={1,2,4,8,16,32,64},
        % xticklabels from table={\testdata}{T},
        xmode=log,
        log basis x={2},
        ymin=0, % Adjust based on your data
        legend pos=north west,
        ymajorgrids=true,
        xmajorgrids=true,
        grid style=dashed,
        legend style={cells={anchor=west}, legend pos=north west,font=\tiny},
    ]

    % Plot for roadusa
    \addplot[color=GBBlue, mark=*] table [x=T, y=roadusaspeedup] {\timepbfs};
    \addlegendentry{road\_usa}

    % Plot for rggn222s0
    \addplot[color=GBRed, mark=square*] table [x=T, y=rggn222s0speedup] {\timepbfs};
    \addlegendentry{rgg\_n\_2\_22\_s0}

    % Plot for delaunayn24
    \addplot[color=GBGreen, mark=triangle*] table [x=T, y=delaunayn24speedup] {\timepbfs};
    \addlegendentry{delaunay\_n24}

    % Plot for hugebubbles-00020
    \addplot [color=GBYellow, mark=pentagon*] table [x=T, y=hugebubblesspeedup] {\timepbfs};
    \addlegendentry{hugebubbles-00020}

    \end{axis}
    \end{tikzpicture}
        \end{minipage}
        \caption{Execution time and speedup of parallel BFS}
    \end{figure}
\end{document}


